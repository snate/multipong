\section{Background and related work} % TODO: Only background?
% TODO: Check from now on: not sure that we don't use the same acronym in intro
\subsection{The Wi-Fi Direct standard}
Wi-Fi Direct standard enables devices to connect with each other, creating 
networks called \textit{groups}, without requiring a wireless access point. It
could be used for several purposes since devices communicate at typical Wi-Fi
speed \cite{bib:wifiP2pspec}, which varies on the 802.11 standard implemented,
as well as the particular features of the devices and the surroundings.
Moreover, Wi-Fi Direct enables devices from different manufacturers to
communicate seamlessly among themselves.

Usually, a device that supports Wi-Fi Direct, in order to create or join a
group, starts a discovery session in which it may find other unconnected Wi-Fi
Direct devices or Group Owners.
In the former case the device can start the formation of a group, whereas in
the latter it asks to join a group.
During the creation of a group, devices negotiate their roles in order to find
a peer that assumes the role of a \emph{logic} access point called Group Owner
(hereafter \textit{GO}). The other devices, even the ones that do not support
Wi-Fi Direct, may decide to connect to the GO.

More precisely, the discovery phase is performed by probing the network in
order to discover other P2P devices, whilst GOs periodically send out beacon to
be detectable. Then, a device may decide to start a group formation by creating
one: in this phase, it exchanges group credentials and specify some features
(e.g. persistency of the group).

When setting up a new Wi-Fi Direct group, the negotiation of the Group Owner
takes place: each one of the devices declares its own
\texttt{GroupOwnerIntent}, which is a value ranging from 0 (not willing to
become the GO) to 15 (highest inclination to become a Group Owner). If the
intents are equal, then there is also a procedure to execute a tie-break.

\subsection{Wi-Fi Direct in Android}
Since its 4.0 version was released, Android has been supporting peer-to-peer 
(\textit{P2P} from now on) Wi-Fi communication that is compliant with the 
Wi-Fi Direct standard\cite{bib:wifiP2pspec}. 
Hence, recent Android devices can form an ad-hoc network with a $1:n$ topology,
in which a Group Owner is connected to multiple P2P clients (hereafter \textit{NGO}s).

As stated in the standard, the GO is decided after a negotiation phase between
the devices; thus, the same hosts may create a P2P network with different GOs
from time to time.

Thanks to the possibility of using Wi-Fi Direct in Android environments, it is
possible to create Android applications that run on different devices and
communicate using this standard, for example game or chat applications.

The implementation of Wi-Fi Direct in Android presents some issues: first of all
Android does not allow a device to join multiple Wi-Fi Direct groups,
but the standard does \cite{bib:android-wifidirect-limits}. Moreover, Android 
devices must ask the user for the permission to join a group, and this hinders 
automatic creation of Wi-Fi Direct networks.

\subsection{Mobile multiplayer games}
Usually, in a mobile game context, the main requirements are (1)
interactivity, i.e. the delay between the user interaction and the game
response should be as short as possible, (2) consistency, i.e. different
players should see coherent and admissible game states, (3) fairness, i.e.
being able of winning matches regardless of different network conditions, (4)
scalability, i.e. being able to support a large number of players, and (5)
continuity, i.e. the present game session should not be interrupted because
of disconnections, handoffs, or any other mobility-related issue
\cite{bib:interactive-mobile-gaming}.

Despite this field offers hard and challenging issues to researchers, there is
still quite a little academic related work: some researchers studied how
the delay affects the gaming experience \cite{bib:impact-delay-multi},
\cite{bib:factors-multi}, whereas others studied online games in different
types of networks \cite{bib:interactive-mobile-gaming},
\cite{bib:survey-mobile-games}.

Mobile games scenarios present their own sets of problems for real-time
applications. Also, it may be relevant to notice that devices in mobile
multiplayer games often connect to a remote server using either Wi-Fi or 3G/4G.

Another option may be to create local subnets with WLAN or Bluetooth, but
even when promoting a node as the local server it might become a
bottleneck and cause poor gaming experience.
Pure P2P or hybrid solutions offer more possibilities and stable networks,
but usually lack of protection from cheaters
\cite{bib:can-mobile-gaming-be-improved}, \cite{bib:study-mobile-phone-sector}.

\subsection{Casual games}
Casual games are video games which have really simple gameplay and are
targeted to mass audience. In fact, casual games are designed to be played
by users with no special skills and without requiring too much time for both
understanding and playing it \cite{bib:mob-health-casual}.

Originally, casual games were played by users through a web browser, but are
now popular on game consoles and smartphones too. A large number of users still
play in a web browser, but through social networks: the idea of casual gaming
has been indeed mashed up with this recent phenomenon, allowing casual gamers to
play with their friends in these platforms.

Casual games experiments were also made in \cite{bib:ppav-casual},
\cite{bib:li-k-social-casual} and \cite{bib:mob-health-casual}\footnote{though
these works were most about the usefulness (if any) and the approval rating of
this type of games rather than technical reports} but, as these studies
reported, this type of games has not break through either the academic or the
commercial world yet.
