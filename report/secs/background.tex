\section{Background and related work} % TODO: Only background?
% TODO: Check from now on: not sure that we don't use the same acronym in intro
\subsection{Wi-Fi Direct: Background}
Wi-Fi Direct standard enables devices to connect with each other, creating 
networks called groups, without requiring a wireless access point. It could be 
used for several purposes because devices communicate at typical Wi-Fi speed 
\cite{bib:wifiP2pspec}, it on whether the devices are 802.11a, g, or n, as well
as the particular characteristics of the devices and the physical environment.
Moreover, Wi-Fi Direct enables devices from different manufactors to communicate.

Usually, a device that supports Wi-Fi Direct, in order to create or join a group,
start a discovery session in which it may find other unconnected Wi-Fi Direct 
devices or Group Owners. In the second case the device ask to join a group, in the 
first one it can start the formation of a group. During the creation of a group 
devices negotiate their roles, in order to find a devices that assumes the role of 
access point called Group Owner (hereafter \textit{GO}) and other devices, even 
ones that doesn't support Wi-Fi Direct, could connect to the GO. 

\subsection{Wi-Fi Direct in Android: Background}
Since its 4.0 version was released, Android has been supporting peer-to-peer 
(\textit{P2P} from now on) P2P Wi-Fi communication that is compliant with the 
Wi-Fi Direct standard\cite{bib:wifiP2pspec}. 
Hence, recent Android devices can form an ad-hoc network with a $1:n$ topology,
in which a Group Owner is connected to multiple P2P clients (hereafter \textit{NGO}s).

As stated in the standard, the GO is decided after a negotiation phase between
the devices; thus, the same hosts may create a P2P network with different GOs
from time to time.

Thanks to the possibility of using Wi-Fi Direct in Android enviroment, it is 
possible to create Android applications that execute on different devices and 
communicate using this standard. 

