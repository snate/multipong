\section{Background and related work} % TODO: Only background?
% TODO: Check from now on: not sure that we don't use the same acronym in intro
\subsection{The Wi-Fi Direct standard}
Wi-Fi Direct standard enables devices to connect with each other, creating 
networks called \textit{groups}, without requiring a wireless access point. It
could be used for several purposes since devices communicate at typical Wi-Fi
speed \cite{bib:wifiP2pspec}, which varies on the 802.11 standard implemented,
as well as the particular features of the devices and the surroundings.
Moreover, Wi-Fi Direct enables devices from different manufacturers to
communicate seamlessly among themselves.

Usually, a device that supports Wi-Fi Direct, in order to create or join a
group, starts a discovery session in which it may find other unconnected Wi-Fi
Direct devices or Group Owners.
In the former case the device can start the formation of a group, whereas in
the latter it asks to join a group.
During the creation of a group,
devices negotiate their roles in order to find a peer that assumes the role of 
access point called Group Owner (hereafter \textit{GO}). The other devices,
even the ones that do not support Wi-Fi Direct, may decide to connect to the
GO.

\subsection{Wi-Fi Direct in Android}
Since its 4.0 version was released, Android has been supporting peer-to-peer 
(\textit{P2P} from now on) Wi-Fi communication that is compliant with the 
Wi-Fi Direct standard\cite{bib:wifiP2pspec}. 
Hence, recent Android devices can form an ad-hoc network with a $1:n$ topology,
in which a Group Owner is connected to multiple P2P clients (hereafter \textit{NGO}s).

As stated in the standard, the GO is decided after a negotiation phase between
the devices; thus, the same hosts may create a P2P network with different GOs
from time to time.

Thanks to the possibility of using Wi-Fi Direct in Android environment, it is 
possible to create Android applications that execute on different devices and 
communicate using this standard, for example game or chat applications. 

The implementation of Wi-Fi Direct in Android presents some issues. First of all
Android doesn't allow to a device to join two different Wi-Fi Direct group, but 
the standard permit that. Another problem is that in Android devices must ask 
to the user the permission to join to a group, and that limit the possibility
of creating networks of devices using Wi-Fi Direct automatically.

\subsection{Casual games}
Casual games are video games which have really simple gameplay and they are
thought to be targeted to mass audience. In fact, casual game users, usually, are
older than other kind of gamers and more predominantly woman. That kind of video
games could be played by users with no special skills and not spending too much 
time.

Originally, casual games were played by users through a web browser, but now 
are popular on game consoles and smartphones.

Nowadays the idea of casual gaming is mashed up with the idea of social networks,
allowing players to play on social network or in multi player mode.