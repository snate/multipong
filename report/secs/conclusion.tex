\section{Conclusion}
%%*************************************************************************
In this paper we have presented issues and a possible solution we
found out while developing a simple multiplayer game based on the Wi-Fi direct
technology. The main problem we met is that the connection between
different peers with Wi-Fi direct is really cumbersome: in fact, this must be
done using the Android settings and very often it takes too much time.

Another problem is managing the unreachability of the GO in a Wi-Fi direct
network: at the current state of the art, the best solution for us is the one
we implemented, i.e. interrupting the current match, because of the
impossibility to recreate the network in a short amount of time and
automatically.

A further issue that we noticed is that the Wi-Fi direct connection
between two different devices could be easily disturbed by trying to connect
one of the two with a third device. Most of the times, the only option is to
cancel the request of connection and restart the formation of the network from
scratch.

On the other hand, if the network is created correctly and there are not
some other devices that try to join the network, we found out that the game
is enjoyable and the Wi-Fi direct could be a quite good solution for
multiplayer mobile games, provided that the number of players is not so big and
the devices used to play are both similar to each other and equipped with good
networking hardware.

There is clearly room for improvements and future work, for instance
using Bluetooth for the networking layer of our application (and performing a
comparison with our current solution) or changing the logical architecture, for
instance setting up a server in the same LAN players are connected to and
play within a local network.