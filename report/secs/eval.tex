\section{Experimental evaluation}

In this section, we evaluate through different tests some interesting metrics related to network activity.

\subsection{TCP and UDP packet RTT and payload}

We use TCP for message exchange during game formation phase, thus an interesting metric lies in application packet's latency and payload.
 
To measure the RTT we decided to send a test-packet to a host which send it back to the sender, then we obtain time delta between message start and delivery. This should be an appropriate approximation of the RTT. This method is affected by a little error due to network state variability and interferences, so we decided to take several measures and then to average them.

For the TCP payload evaluation, we generated some logs in the application to get the exact size of each type of message [reference to messages type]. Messages carry always the same data structure and length, so in this case it is not necessary to calculate a mean value.

As for the TCP packets, we are interested in UDP RTT and payload statistics, so we measured them with the same approaches.

\subsection{WiFi traffic}

A more interesting evaluation is about general traffic over WiFi. The application was projected to minimize message exchange among peers, so we were interested to know how much traffic our application generates trough the WiFi. To determine that, we decided to use an external monitoring application, \textit{3G Watchdog} (3GW from now on) [add reference], but there is a plenty of alternatives available over the market. This application is intended to monitor long-time traffic such as mobile connection over month/week, but it also keeps track of the time the application has run, so we were able to calculate both traffic and mean bandwidth allocation.

Multipong, anyway, generates different amounts of traffic in different phases; during the game creation phase the traffic is way more variable than in the game phase, mainly due to:

\begin{itemize}
\item number of peers
\item WiFi interferences
\end{itemize}

So, we decided to take measures of several game matches and then to average them to get a minimal variation, applying the following workflow:

\begin{itemize}
\item For each new match
	\begin{itemize}
	\item Start 3GW and reset its data counters
	\item Start a new game and play, possibly trying to variate match duration
	\item Annotate game transferred data and game duration
	\end{itemize}
\end{itemize}

This approach made us able to minimize the variation due to interferences, but the number of peers still interferes with these measurements. So, we decided to group them by number of peers; more precisely we chose to make 2, 3 and 4 player matches. The comparison among these groups reveals the influence of the pairing phase to the traffic.

\subsection{Battery Consumption}

Usually, apps which do a massive use of WiFi are also highly energy-consuming. One of our minor purposes was to optimize the battery-life of the devices running our application, so we decided to track energy consumption over the game. We identified a nice tool for energy monitoring in \textit{PowerTutor} (PT hereafter), a free app available on Play Store widely used by the Android Development community. Clearly, this kind of applications are unable to do physical measures of the energy consumption, they just estimate it through a mathematical model created over results from old Android devices. 

This approach is however considered quite reliable for that cases where energy monitoring is not the key factor to consider for the experiment. Nevertheless, because of our communication minimization approach, we decided to compare energy consumption in multiplayer mode against the singleplayer mode, and because of different traffic on GOs and NGOs we also decided to consider these two scenarios.
