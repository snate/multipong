\section{Introduction}
Thanks to the diffusion of smartphones, the mobile games market is becoming
more and more important. In 2016 the global games market reached \$99.6
billion and for the first time mobile gaming took a larger share than PC
with \$36.9 billion (up to 21.3\% globally), reaching the 37\% of the game
market \cite{bib:newzoo2}. Also, mobile games market is expanding so much that
in 2019 it is predicted to reach even half of all the global games market
\cite{bib:newzoo}. This fact leads us to think that it is worth studying
this emergent field, also because smartphones and tablets are nowadays
pervasive in everyday's life of people.

Mobile games are more challenging than console or PC games because of the
limited resources available on the devices. In fact, battery energy and
mobile data transfer savings are critical. Another issue a developer may
encounter when dealing with mobile devices could be the input mechanism:
PCs and consoles allow users to play using mouses, keyboards and joysticks,
whereas with smartphones or tablets games can be played (mostly) using the
touch sensor (soft keypad) \cite{bib:mobile-input-devices}. Moreover, mobile
games are usually played within screens that are smaller than a PC monitor or a
TV.

Some mobile games support multiplayer features. There are indeed several
options, for instance live synchronous games, in which players are matched
together to compete, or turn-based asynchronous games, in which local actions
of a player are recorded and than broadcasted to the other participants.

If you visit Google's Play Store, you might notice that most of the top
grossing games are online multiplayer games, with many of them requiring
real-time users interaction. Clearly, since pleasant online real-time gaming
was already hard to achieve within the context of \textit{wired} networks, now
it is even more difficult with \textit{wireless} networks and devices with
power saving issues.

Moreover, things get even tougher when it comes to ad-hoc networks and mobile
devices: most of the games you can find on Google's Play Store are indeed
multiplayer games that work with a remote server. Furthermore, the few mobile
games we were able to find for local networking are limited to two-players
matches. Hence, we are interested in finding out how a simple Android
application would perform in an ad-hoc network with matches composed of
possibly more than two players.

Even if the quality of experience (\textit{QoE}) is a broad term that is
comprehensive of many factors\cite{bib:moeller-qoe}, our aim is to focus on
some technical aspects that affects the interaction quality: in particular, we
want to measure the generated network traffic and the battery consumption for
such a game.

The remainder of the paper is organized as follows: in section II we talk about
other work in this context and talks about ad-hoc networking support in Android;
in section III we describe how we developed the game; in Section IV we discuss
how we conducted the experiments; in Section V we show the results of our tests;
finally, Section VI wraps up this paper.
