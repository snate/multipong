\section{Multipong} % TODO: Better title?

In this section we will describe how we developed the Multipong application,
focusing on some architectural and implementation choices we made to improve
the game quality.

\subsection{Game description} % TODO: Better title?

Multipong is a tribute to the Pong game, one of the first arcade videogames.
Instead of playing against an AI, there are a single-player mode and a
multiplayer mode.

In the single-player mode the player scores a point each time the paddle hits
the ball, making it bounce upwards until it reaches the top edge and then the
ball falls down again. Clearly, the player loses the game when the paddle misses
the ball.

In the multiplayer mode, several human players connect their devices in order
to form an ad hoc network and then when one of the players hits the ball, it is
transferred to the next player's screen as if their gameboards were joint.
When a player misses the ball, he loses a life and the ball is thrown out
randomly to the next player's screen.
If a player runs out of lives, she will not be able to play for the rest of the
game.

Before starting a multiplayer game a player (called the \textit{host}) creates
a match, allowing other players in his same network to join. Then, when the
game starts, the host will be the initial player.
